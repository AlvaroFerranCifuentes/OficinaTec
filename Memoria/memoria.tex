\newpage
%%%%%%%%%%%%%%%%%%%%%%%%%%%%%%%%%%%%%%%%%%%%%%%%%%%%%%%%%%%%%%%%%%%%%%%%%%%%%%%%%%%%%%%%%%%%%%%%%%%%%%%%%%%%%%%%%%%%%
\section{Objetivo}
El objetivo del presente proyecto es definir y justificar los datos y características constructivas y técnicas necesarios para la realización de la instalación eléctrica automatización de una fábrica de complementos alimenticios, exponiendo ante los organismos competentes que la instalación que nos ocupa reúne las condiciones y garantías exigidas por la reglamentación vigente, a fin de obtener la Autorización Administrativa y la de Ejecución de la instalación por parte de la Dirección General de Industria, Energía y Minas, Gobierno de la Comunidad de Madrid y Ayuntamiento de Rivas-Vaciamadrid.\
 
%%%%%%%%%%%%%%%%%%%%%%%%%%%%%%%%%%%%%%%%%%%%%%%%%%%%%%%%%%%%%%%%%%%%%%%%%%%%%%%%%%%%%%%%%%%%%%%%%%%%%%%%%%%%%%%%%%%%%
\section{Ubicacion}

La Instalación objeto del presente proyecto se realizará en una nave industrial de $1620m^2$ de superficie útil construida en el año 1992 ubicada en la siguiente dirección:\\

 {\bfseries Calle de la Fundición 8, Rivas-Vaciamadrid, Comunidad de Madrid, (España)}

%%%%%%%%%%%%%%%%%%%%%%%%%%%%%%%%%%%%%%%%%%%%%%%%%%%%%%%%%%%%%%%%%%%%%%%%%%%%%%%%%%%%%%%%%%%%%%%%%%%%%%%%%%%%%%%%%%%%%
\section{Antecedentes}

La nave industrial en la cual se realizará la instalación objeto de este proyecto estuvo ocupada por un concesionario y taller de reparación de automóviles desde el año 1992 hasta el año 2013. La propiedad adquirió la nave industrial en el año 2013 con el objetivo de reconvertir el espacio en una fábrica de complementos alimenticios, dada la fuerte demanda que está teniendo este tipo de productos en los últimos años. Las instalaciones eléctricas de se desmantelaron para la reconversión, siendo su diseño parte del objeto del presente proyecto.\

%%%%%%%%%%%%%%%%%%%%%%%%%%%%%%%%%%%%%%%%%%%%%%%%%%%%%%%%%%%%%%%%%%%%%%%%%%%%%%%%%%%%%%%%%%%%%%%%%%%%%%%%%%%%%%%%%%%%%
\section{Normativa}

En el estudio y redacción del siguiente proyecto se han tenido en cuenta las siguientes normas y reglamentos actualmente en vigor:\\

{\bfseries Ley 54/1997 de 27 de Noviembre, de Regulación del Sector Eléctrico} (B.O.E. 28 de Noviembre de 1997).\\

{\bfseries Real Decreto 1995/2000}, de 1 de diciembre, por el que se regulan las actividades de transporte, distribución, comercialización, suministro y procedimientos de autorización de instalaciones de energía eléctrica.\\

{\bfseries Reglamento Electrotécnico para Baja Tensión. (R.E.B.T.)}\
Real Decreto 842/2002 de 2 de agosto.\\

{\bfseries Orden de 13-03-2002 de la Consejería de Industria y Trabajo}, por la que se establece el contenido mínimo en proyectos de industrias y de instalaciones Industriales.\\

{\bfseries Código Técnico de la Edificación.} (C.T.E.) Real Decreto 314/2006, de 17 de marzo.\\

{\bfseries Normas UNE, UNE-EN, EN e IEC}\\

{\bfseries Normas particulares de la Compañía Suministradora.}\\

{\bfseries Ordenanzas Municipales del Ayuntamiento de Rivas-Vaciamadrid.}\\

{\bfseries Directiva 2002/46/CE} del Parlamento Europeo y del Consejo, de 10 de junio de 2002, relativa a la aproximación de las legislaciones de los Estados miembros en materia de complementos alimenticios.\\

{\bfseries Reglamento (CE) 1137/2008} del Parlamento Europeo y del Consejo, de 22 de octubre de 2008 , por el que se adaptan a la Decisión 1999/468/CE del Consejo determinados actos sujetos al procedimiento establecido en el artículo 251 del Tratado, en lo que se refiere al procedimiento de reglamentación con control.\\

{\bfseries Reglamento (CE) n o 1170/2009} de la Comisión, de 30 de noviembre de 2009 , por la que se modifican la Directiva 2002/46/CE del Parlamento Europeo y del Consejo y el Reglamento (CE) n o 1925/2006 del Parlamento Europeo y del Consejo en lo relativo a las listas de vitaminas y minerales y sus formas que pueden añadirse a los alimentos, incluidos los complementos alimenticios.\\

{\bfseries Real Decreto 1487/2009}, de 26 de septiembre, relativo a los complementos
alimenticios.\\

{\bfseries Otros reglamentos vigentes que le sean de aplicación.}\\

En caso de producirse alguna diferencia de criterio entre una normativa y otra debe permanecer la de rango superior, siempre y cuando la de rango inferior no fuese más perceptiva.

%%%%%%%%%%%%%%%%%%%%%%%%%%%%%%%%%%%%%%%%%%%%%%%%%%%%%%%%%%%%%%%%%%%%%%%%%%%%%%%%%%%%%%%%%%%%%%%%%%%%%%%%%%%%%%%%%%%%%
\pagebreak

\section{Descripción general}

\subsection{Instalación de maquinaria y automatización y control del proceso de fabricación}

Se realizará la instalación de la maquinaria y automatización y control del proceso de fabricación acorde a las necesidades del proceso productivo. Cada máquina posee una serie de sensores y receptores que se conectarán a través de un bus de datos  a un PC industrial mediante una tarjeta de adquisición de datos. Los robots que paletizarán el producto al final de la línea de producción usarán un protocolo de comunicación DeviceNet y se conectarán al PC industrial mediante una tarjeta interfaz de la misma tecnología. La programación del proceso productivo se realizará a través de un software específico atendiendo a las especificaciones deseadas para el producto.

\subsection{Instalación eléctrica}

Se realizará la instalación eléctrica de la zona de oficina, compuesta por la iluminación, tomas de corriente de uso general y tomas de corriente SAI. En la zona de fabricación y almacén se electrificarán cada uno de las máquinas que participan en el la producción, así como las luminarias y las tomas de corriente de uso general. Se procederá al cálculo de los conductores, protecciones y receptores acorde a las disposiciones mínimas exigidas por las autoridades competentes y los requisitos de cada parte de la instalación, realizándose los cálculos necesarios para justificar las protecciones, calibre de los conductores que alimentan a los receptores y requisitos mínimos de flujo luminoso en las zonas de trabajo.

\pagebreak

%%%%%%%%%%%%%%%%%%%%%%%%%%%%%%%%%%%%%%%%%%%%%%%%%%%%%%%%%%%%%%%%%%%%%%%%%%%%%%%%%%%%%%%%%%%%%%%%%%%%%%%%%%%%%%%%%%% 
\section{Elementos del sistema}
\subsection{Maquinaria}

La maquinaria a instalar en la zona de fabricación se detalla en la siguiente figura: 

\begin{figure}[htp]
			\centering
			\includegraphics[width=0.55\textwidth]{Memoria/planta_maquinas.png}
			\caption{Elementos de la instalación, maquinaria}
			\label{fig:testb}
	\end{figure}

\pagebreak

	\subsubsection{Máquina de secado de polvo Hengli HSA1508-0711NH}

	

	Horno industrial Hengli HSA1508-0711NH de secado de alta temperatura. El interior del horno es capaz de alcanzar temperaturas máximas de 1100ºC, teniendo una temperatura de operación recomendada de 1050ºC.\\

	
	\begin{figure}[htp]
		\begin{minipage}{.48\textwidth}
			\centering
			\includegraphics[scale=0.5]{Datasheets/1HornoFoto.png}
			\caption{Detalle Hengli HSA1508-0711NH}
			\label{fig:testa}
		\end{minipage}
		\begin{minipage}{.48\textwidth}
			\centering
			\includegraphics[scale=0.5]{Datasheets/Miniaturas/horno.png}
			\caption{Plano de planta Hengli HSA1508-0711NH}
			\label{fig:testb}
		\end{minipage}
	\end{figure}

	Por su interior circula una cinta transportadora de 150mm de ancho a una velocidad de entre 30 y 200mm por minuto, siendo regulable según las necesidades de producción.\\

	Tiene unas dimensiones de 7720mm de largo, 1200mm de ancho y 1350mm de altura máxima. La altura de túnel es de 80mm y está compuesto por una mezcla de gases $N_2$ y $H_2$. Tiene una uniformidad de temperatura de $\pm$2 $\degree C$ a lo largo de todo el horno para garantizar un secado regular y correcto.\\

	Pesa 1100kg y tiene un panel de control para monitorizar y actuar sobre los procesos directamente, si fuera necesario tomar el control manual.

		\begin{itemize}
				\item{Conexionado mecánico:}
				
				Se fijará al suelo con 14 pies de máquina antivibración y de nivelación de 120mm de diámetro.

				\item{Conexionado eléctrico:}

				Máquina trifásica con demanda de potencia de 4kW. Se conectará al punto de alimentación de la máquina mediante una  toma macho IEC 60309. En el caso de que la máquina venga desprovista de el cable de alimentación, se conectará al bornero un cable multipolar con las mismas características que el cable de alimentación al punto de alimentación de máquina como se especifica en el el [APARTADO PLANOS], dejando un exceso de cable de 1m. \
				
				\item{Comunicación de datos:}

				Se conecta a la tarjeta de adquisición de datos NET-26C mediante un cable de red RJ-45.
				
		\end{itemize}

	\newpage

	\subsubsection{Máquina de compresión en tabletas Unimach YUJ-17BZ}

	

	Máquina industrial de formación de comprimidos Unimach YUJ-17B2. Transforma polvo seco en pastillas de diámetro de entre 3mm y 12mm y de cualquier forma para la que se introduzca el correspondiente molde.\\

	\begin{figure}[htp]
		\begin{minipage}{.48\textwidth}
			\centering
			\includegraphics[scale=1]{Datasheets/2MaquinaPrensadoFoto.png}
			\caption{Detalle U. YUJ-17BZ}
			\label{fig:testa}
		\end{minipage}
		\begin{minipage}{.48\textwidth}
			\centering
			\includegraphics[scale=0.8]{Datasheets/Miniaturas/compresion.png}
			\caption{Plano de planta U. YUJ-17BZ}
			\label{fig:testb}
		\end{minipage}
	\end{figure}
		

	Es capaz de ejercer una presión máxima de 100kN, con una fase de pre-presionado a 16kN. Es capaz de procesar entre 5 y 50 kilogramos de pastillas por hora. \\

	Tiene unas dimensiones de 2820mm de largo, 1530mm de ancho y 1540mm de alto. Pesa 500kg y tiene una potencia de motor de 4kW, .\\

	Lleva incorporado un controlador PLC de alta velocidad así como una interfaz de usuario para operación manual en caso de necesidad. La totalidad de la máquina está realizada con acero inoxidable. 


		\begin{itemize}
				\item{Conexionado mecánico:}
				
				Se fijará al suelo con 4 pies de máquina antivibración y de nivelación de 80mm de diámetro.

				\item{Conexionado eléctrico:}

				Máquina trifásica con demanda de potencia de 2kW. Se conectará al punto de alimentación de la máquina mediante una  toma macho IEC 60309. En el caso de que la máquina venga desprovista de el cable de alimentación, se conectará al bornero un cable multipolar con las mismas características que el cable de alimentación al punto de alimentación de máquina como se especifica en el el [APARTADO PLANOS], dejando un exceso de cable de 1m. \
								
				\item{Comunicación de datos:}

				Se conecta a la tarjeta de acquisición de datos NET-26C mediante un cable de red RJ-45.
		\end{itemize}

	\newpage

	\subsubsection{Máquina de verificación de dureza de tabletas Unimach YJX220B}

		
	Máquina industrial de inspección de pastillas Unimach YJX220B. Integra diferentes tecnologías para cumplir con los requerimientos del sector farmacéutico. Puede comprobar la dureza de tabletas y encapsulados, sean estos duros o blandos.\\

	\begin{figure}[htp]
		\begin{minipage}{.48\textwidth}
			\centering
			\includegraphics[scale=0.6]{Datasheets/3Foto.png}
			\caption{Detalle Unimach YJX220B}
			\label{fig:testa}
		\end{minipage}
		\begin{minipage}{.48\textwidth}
			\centering
			\includegraphics[scale=0.8]{Datasheets/Miniaturas/verificacion.png}
			\caption{Plano de planta Unimach YJX220B}
			\label{fig:testb}
		\end{minipage}
	\end{figure}
	

	Los productos entran en el feeder a través de un embudo y son limpiados por un flujo de aire a presión que elimina restos de polvo. Después son colocados en fila para comprobar que resisten un umbral de presión. También se puede regular para comprobar periódicamente que se pueden deshacer por debajo de un umbral máximo de presión, lo cual indica que las pastillas no se desharán durante el transporte pero podrán ser digeridas por los consumidores.\\


	Es capaz de testear hasta 120 000 cápsulas por hora, o aproximadamente 50kg de pastillas dependiendo de la forma, tamaño y peso de las mismas.\\



		\begin{itemize}
				\item{Conexionado mecánico:}
				
				Se fijará al suelo con 6 pies de máquina antivibración y de nivelación de 80mm de diámetro.

				\item{Conexionado eléctrico:}

				Máquina trifásica con demanda de potencia de 1,2kW. Se conectará al punto de alimentación de la máquina mediante una  toma macho IEC 60309. En el caso de que la máquina venga desprovista de el cable de alimentación, se conectará al bornero un cable multipolar con las mismas características que el cable de alimentación al punto de alimentación de máquina como se especifica en el el [APARTADO PLANOS], dejando un exceso de cable de 1m. \
				
				\item{Comunicación de datos:}

				Se conecta a la tarjeta de adquisición de datos NET-26C mediante un cable de red RJ-45.
		\end{itemize}

	\newpage


	\subsubsection{Máquina de revestimiento de tabletas Unimach BGB80}
		
	
	
	Máquina industrial de revestimiento de tabletas Unimach BGB80. Su función es revestir las pastillas de una capa de agentes específicos para darles un color, sabor o textura específicos, así como endurecer la parte exterior de la misma para evitar que se deshaga hasta su uso.\\

	\begin{figure}[htp]
		\begin{minipage}{.48\textwidth}
			\centering
			\includegraphics[scale=0.3]{Datasheets/4Foto.png}
			\caption{Detalle Unimach BGB80}
			\label{fig:testa}
		\end{minipage}
		\begin{minipage}{.48\textwidth}
			\centering
			\includegraphics[scale=0.6]{Datasheets/Miniaturas/revestimiento.png}
			\caption{Plano de planta Unimach BGB80}
			\label{fig:testb}
		\end{minipage}
	\end{figure}

	
	

	Las pastillas entran en una cinta hasta un depósito regulado. Una vez el sensor detecte que el contenido ha llegado al peso establecido este vuelca las pastillas a un tambor giratorio para que comience el proceso.\\

	Una pistola de pintura a una presión de entre 0.29 MPa y 0.39 MPa y rotando 360 grados libera ráfagas de particulas de la solución. El tambor al girar a una velocidad 5rpm hace rodad las pastillas y se asegura de que queden todas uniformemente cubiertas.\\

		\begin{itemize}
				\item{Conexionado mecánico:}\\
				
				Se fijará al suelo con 12 pies de máquina antivibración y de nivelación de 80mm de diámetro.

				\item{Conexionado eléctrico:}

				Máquina trifásica con demanda de potencia de 4kW. Se conectará al punto de alimentación de la máquina mediante una  toma macho IEC 60309. En el caso de que la máquina venga desprovista de el cable de alimentación, se conectará al bornero un cable multipolar con las mismas características que el cable de alimentación al punto de alimentación de máquina como se especifica en el el [APARTADO PLANOS], dejando un exceso de cable de 1m. \
				
				\item{Comunicación de datos:}

				Se conecta a la tarjeta de adquisición de datos NET-26C mediante un cable de red RJ-45.
		\end{itemize}

    \newpage

	\subsubsection{Máquina de sellado de blisters Unimach DPK-260H2}

	
	Máquina industrial de formación y sellado de blisters Unimach DPK-260H2. Su función es agrupar y encapsular las pastillas en láminas de plástico con cubierta de aluminio para facilitar su organización, distribución y almacenamiento, así como evitar que entren en contacto con agentes contaminantes hasta el momento de su uso.\\


	\begin{figure}[htp]
		\begin{minipage}{.48\textwidth}
			\centering
			\includegraphics[scale=0.4]{Datasheets/5Foto.png}
			\caption{Detalle U. DPK-260H2}
			\label{fig:testa}
		\end{minipage}
		\begin{minipage}{.48\textwidth}
			\centering
			\includegraphics[scale=0.4]{Datasheets/Miniaturas/blister.png}
			\caption{Plano de planta U. DPK-260H2}
			\label{fig:testb}
		\end{minipage}
	\end{figure}
	
	

	Las pastillas entran por una cinta a un acumulador. Una vez lleno este las deja caer a una matriz para colocarlas en posición. A la vez un cabezal móvil caliente funde el plástico en los lugares especificados para crear los receptáculos donde se colocan las pastillas. Tras ser depositadas una lámina de aluminio con los detalles correspondientes ya impresos cubre el conjunto y se aplica calor para sellarlo. Finalmente se corta en tiras del tamaño deseado para su posterior empaquetado. Tiene una capacidad de procesado de hasta 140 000 pastillas por hora, y una frecuencia de corte de entre 25 y 60 veces por minuto. La profundidad máxima de los receptáculos es de 22mm, utilizando una lámina de plástico de entre 0.25mm y0.5mm de grosor. La lámina de aluminio tiene un grosor de entre 0.02mm y0.035mm.


		\begin{itemize}
				\item{Conexionado mecánico:}
				
				Se fijará al suelo con 10 pies de máquina antivibración y de nivelación de 80mm de diámetro.

				\item{Conexionado eléctrico:}

				Máquina trifásica con demanda de potencia de 4kW. Se conectará al punto de alimentación de la máquina mediante una  toma macho IEC 60309. En el caso de que la máquina venga desprovista de el cable de alimentación, se conectará al bornero un cable multipolar con las mismas características que el cable de alimentación al punto de alimentación de máquina como se especifica en el el [APARTADO PLANOS], dejando un exceso de cable de 1m. \
				
				\item{Comunicación de datos:}

				Se conecta a la tarjeta de adquisición de datos NET-26C mediante un cable de red RJ-45.
		\end{itemize}
	
	\newpage

	\subsubsection{Máquina de empaquetado Huan Yuan HTZ200}

	

	Máquina industrial de empaquetado Huan Yuan HTZ200. Su función es formar cajas a partir de los cartones ya impresos introducidos en la máquina e introducir los paquetes de pastillas en los mismos. También debe plegar los folletos de instrucciones e insertarlos en las cajas antes de cerrarlas. \\

	\begin{figure}[htp]
		\begin{minipage}{.48\textwidth}
			\centering
			\includegraphics[scale=0.4]{Datasheets/7Foto.png}
			\caption{Detalle Huan Yuan HTZ200}
			\label{fig:testa}
		\end{minipage}
		\begin{minipage}{.48\textwidth}
			\centering
			\includegraphics[scale=0.5]{Datasheets/Miniaturas/empaquetar.png}
			\caption{Plano de planta Huan Yuan HTZ200}
			\label{fig:testb}
		\end{minipage}
	\end{figure}

	

	La máquina agarra los cartones que carga el operario y los dobla formando la caja diseñada. Acto seguido agarra una hoja de folleto y la pliega, introduciéndola en la caja para finalmente insertar el blister y cerrar la caja.\\

	Tiene una capacidad de empaquetado de 100 a 200 cajas por minuto, pudiendo formar cajas con dimensiones de 55mm a 180mm de largo por 35mm a 85mm de ancho por 14mm a 60mm de alto.


		\begin{itemize}
				\item{Conexionado mecánico:}
				
				Se fijará al suelo con 6 pies de máquina antivibración y de nivelación de 80mm de diámetro.

				\item{Conexionado eléctrico:}

				Máquina trifásica con demanda de potencia de 4kW. Se conectará al punto de alimentación de la máquina mediante una  toma macho IEC 60309. En el caso de que la máquina venga desprovista de el cable de alimentación, se conectará al bornero un cable multipolar con las mismas características que el cable de alimentación al punto de alimentación de máquina como se especifica en el el [APARTADO PLANOS], dejando un exceso de cable de 1m. \
				
				\item{Comunicación de datos:}

				Se conecta a la tarjeta de adquisición de datos NET-26C mediante un cable de red RJ-45.
		\end{itemize}

\newpage


	\subsubsection{Cintas transportadoras TEMPO}

	

	Cintas industriales de transporte TEMPO. Su función será la de llevar los diferentes elementos de la cadena de producción entre las diferentes máquinas que la componen. \\

	\begin{figure}[htp]
		\begin{minipage}{.48\textwidth}
			\centering
			\includegraphics[scale=0.15]{Datasheets/11Foto2.png}
			\caption{Detalle de Cinta TEMPO}
			\label{fig:testa}
		\end{minipage}
		\begin{minipage}{.48\textwidth}
			\centering
			\includegraphics[scale=1]{Datasheets/Miniaturas/cinta.png}
			\caption{Plano de planta Cinta TEMPO}
			\label{fig:testb}
		\end{minipage}
	\end{figure}
	
	

	Están cubiertas por una cinta de polyester y plástico, de superficie acolchada para facilitar el transporte de los elementos, exceptuando la que transporta el polvo hasta la máquina de compresión, que es lisa para que este se pueda soltar de la cinta. \\

	Todas las cintas son aptas para transporte de componentes destinados a consumo humano. 

		\begin{itemize}
				\item{Conexionado mecánico:}
				
				Se fijarán al suelo con los suficientes pies de máquina antivibración y de nivelación de diámetro suficientes para cada cinta, de forma que quede fija y no vibre durante su normal funcionamiento y teniendo en cuenta las diferentes longitudes y pesos de cada una.

				\item{Conexionado eléctrico:}

				Máquina trifásica con demanda de potencia de 4kW. Se conectará al punto de alimentación de la máquina mediante una  toma macho IEC 60309. En el caso de que la máquina venga desprovista de el cable de alimentación, se conectará al bornero un cable multipolar con las mismas características que el cable de alimentación al punto de alimentación de máquina como se especifica en el el [APARTADO PLANOS], dejando un exceso de cable de 1m. \ 
				
				\item{Comunicación de datos:}

				Se conecta a la tarjeta de adquisición de datos NET-26C mediante un cable de red RJ-45.
		\end{itemize}


\newpage

\subsubsection{Robot de paletizado ABB IRB 260}

	
	
	Para las tareas de paletizado de las cajas del producto a la salida de la línea de producción se instalará un robot ABB IRB 260 especializado en tareas de picking, con un controlador ABB IRC5, con módulo de entradas y salidas digitales. La comunicación con el PC industrial la realizará a través de un bus de datos DeviceNet. La programación del robot se realizó con el sotware ABB RobotStudio 5.12. El código fuente se encuentra en el [ANEXO QUE SEA]\\
	
	\begin{figure}[htp]
		
			\centering
			\includegraphics[scale=0.3]{Datasheets/8Foto.jpg}
			\caption{Robot ABB IRB 260 y controlador IRC5}
			\label{fig:testa}
		
	\end{figure}

		\begin{itemize}
				\item{Conexionado mecánico:}
				
				Se fijará al suelo con 3 pies de máquina antivibración y de nivelación de 80mm de diámetro.

				\item{Conexionado eléctrico:}

				Máquina trifásica con demanda de potencia de 4kW. Se conectará al punto de alimentación de la máquina mediante una  toma macho IEC 60309. En el caso de que la máquina venga desprovista de el cable de alimentación, se conectará al bornero un cable multipolar con las mismas características que el cable de alimentación al punto de alimentación de máquina como se especifica en el el [APARTADO PLANOS], dejando un exceso de cable de 1m. \	
							
				\item{Comunicación de datos:}

				Se conecta a la tarjeta de interfaz DeviceNet CAN-AC1-PCI/DN mediante un cable DeviceBus de Clase 2 18 AWG trenzado.
		\end{itemize}
\newpage

	\subsubsection{Robot de posicionamiento de palets ABB IRB 4400}

	
	
	Para las tareas de posicionamiento de los palets vacíos en la zona de paletizado se instalará un robot ABB IRB 4400, con capacidad de carga de 20kg, con un controlador ABB IRC5, con módulo de entradas y salidas digitales. La comunicación con el PC industrial la realizará a través de un bus de datos DeviceNet. La programación del robot se realizó con el sotware ABB RobotStudio 5.12. El código fuente se encuentra en el [ANEXO QUE SEA]\\
	

	\begin{figure}[htp]
		
			\centering
			\includegraphics[scale=0.4]{Datasheets/9Foto.jpg}
			\caption{Robot ABB IRB 4400 y controlador IRC5}
			\label{fig:testa}
		
	\end{figure}	
	

		\begin{itemize}
				\item{Conexionado mecánico:}
				
				Se fijará al suelo con 3 pies de máquina antivibración y de nivelación de 80mm de diámetro.

				\item{Conexionado eléctrico:}

				Máquina trifásica con demanda de potencia de 4kW. Se conectará al punto de alimentación de la máquina mediante una  toma macho IEC 60309. En el caso de que la máquina venga desprovista de el cable de alimentación, se conectará al bornero un cable multipolar con las mismas características que el cable de alimentación al punto de alimentación de máquina como se especifica en el el [APARTADO PLANOS], dejando un exceso de cable de 1m. \	
							
				\item{Comunicación de datos:}

				Se conecta a la tarjeta de interfaz DeviceNet CAN-AC1-PCI/DN mediante un cable DeviceBus de Clase 2 18 AWG trenzado.
		\end{itemize}
\newpage

	\subsubsection{Tarjeta interfaz DeviceNet CAN-AC1-PCI/DN}

	
		Tarjeta interfaz DeviceNet CAN-AC1-PCI/DN con microcontrolador para procesado de datos embebido. Su función es gestionar la comunicación entre el PC industrial y el controlador de los robots ABB. Se conectará a ambos mediante un cable DeviceBus de Clase 2 18 AWG trenzado mediante una instalación bajo bandeja salvacables. Interfaz PCI universal, compatible con el protocolo CAN V2.0 (11/29 Bit IDs), para el conexionado del cableado posee un bornero de 5 pines y es  compatible con los sistemas operativos Windows Vista, Windows 2000, Windows XP y Linux.\\

		\begin{figure}[htp]
			\centering
			\includegraphics[width=1\textwidth]{Datasheets/10aFoto.jpg}
			\caption{Tarjeta DeviceNet CAN-AC1-PCI/DN}
			\label{fig:testa}
	\end{figure}

		\begin{itemize}
				\item{Conexionado mecánico:}
				
				Se insertará en la bahía PCI del PC industrial y se fijará mediante tornillos estándar.

				\item{Conexionado eléctrico:}

				El PC industrial proporcionará la alimentación necesaria para su funcionamiento a través del puerto PCI.
 				
				\item{Comunicación de datos:}
				
				Se conectará a ambos robots ABB mediante un cable DeviceBus de Clase 2 18 AWG trenzado mediante una instalación bajo bandeja salvacables.

		\end{itemize}

\newpage


	\subsubsection{Tarjeta de adquisición de datos Eagle Technologies NET-26C}

	
		Tarjeta de adquisición de datos Eagle Technologies NET-26C. Se encarga de realizar la función de puente entre la maquinaria utilizada y el ordenador industrial. Se conecta mediante cable de red RJ-45 a todas las máquinas, que poseen un puerto para el mismo, recibiendo información de los diversos sensores y gestionando la activación de los actuadores de las máquinas.\\

		\begin{figure}[htp]
			\centering
			\includegraphics[width=0.6\textwidth]{Datasheets/10Foto.jpg}
			\caption{Tarjeta Eagle Technologies NET-26C}
			\label{fig:testa}
	\end{figure}


		Tiene 16 canales de entrada analógicos y digitales, con una resolución de 16 bits en modo analógico y una frecuencia de refresco de 250kHz. Posee su propia fuente de alimentación de 9V DC y es compatible con los sistemas operativos Windows Vista, Windows 2000, Windows XP y Linux.\\

		\begin{itemize}
				\item{Conexionado mecánico:}
				
				Se fijará a la pared con seis tornillos simples de 50mm de longitud y 6mm de diámetro o superiores.

				\item{Conexionado eléctrico:}

				Se conectará la fuente de alimentación de 9V DC mediante un cable de alimentación simple a una toma de corriente de uso general.
				 				
				\item{Comunicación de datos:}
				
				Se conecta al PC industrial Siemens mediante un cable de red RJ-45.

		\end{itemize}

\newpage

	\subsubsection{PC industrial Siemens IPC547E}


	PC industrial de Siemens SIMATIC IPC547E encargado de controlar todo el proceso de fabricación, gestionando la información recibida por la tarjeta de adquisición sobre los sensores de las diferentes máquinas en tiempo real y proporcionando instrucciones a los diferentes actuadores. Posee un sistema operativo Linux Debian 6 que cumple con el estándar POSIX. Se podrá controlar y monitorizar desde los ordenadores de la sala de control.\\

	\begin{figure}[htp]
			\centering
			\includegraphics[scale=0.4]{Datasheets/12Foto.png}
			\caption{Tarjeta Eagle Technologies NET-26C}
			\label{fig:testa}
	\end{figure}
	

		\begin{itemize}
				\item{Conexionado mecánico:}
				
				Se introducirá en el armario rack destinado a tal efecto en la oficina, cerrándose la puerta para evitar manipulaciones no autorizadas y riesgos de golpes.

				\item{Conexionado eléctrico:}
				
				La alimentación se realizará mediante toma de corriente del circuito de SAI.

				\item{Comunicación de datos:}
				
				Se conecta a la tarjeta de adquisición de datos NET-26C mediante un cable de red RJ-45.
		\end{itemize}

\newpage

\subsubsection{Conexiones máquinas}
\hfill
	\begin{figure}[htp]
			\centering
			\includegraphics[scale=0.35]{Planos/conexiones.png}
			\caption{Diagrama de conexiones entre máquinas}
			\label{fig:testa}
	\end{figure}

\hfill.
\newpage
	
\subsection{Instalación eléctrica}

\subsubsection{Red de suministro}

La empresa encargada de suministrar energía eléctrica a la instalación será:\\
UNIÓN FENOSA, S.A.\\

Características de la red de suministro:\

\begin{itemize}
\item {\bfseries Red:} Corriente Alterna Trifásica 
\item {\bfseries Tensión:} 400/230V (Entre fases y entre fase y neutro)
\item {\bfseries Frecuencia:} 50 Hz
\item {\bfseries Intensidad de cortocircuito trifásico:} 12 kA
\end{itemize}

La instalación de suministro será existente. Se realizará a través de {\bfseries Acometida Subterránea con conductores unipolares de aluminio RV 0,6/1kV 3x240+1x150 Al} instalados bajo tubo enterrado, llegando a una CGP de instalación empotrada. , siendo sólo objeto de este proyecto la instalación aguas abajo a partir del Cuadro General de Mando y Protección cuya envolvente se ajusta a las normas UNE 20.451 y UNE-EN 60.439 -3, con un grado de protección IP 30 según UNE 20.324 e IK07 según UNE-EN 50.102.\\

Para la sección de conductores indicada previamente, teniendo en cuenta sus características y que los fusibles instalados en la CGP son de 350A, se establece una {\bfseries potencia máxima admisible en la instalación de 243kW}

\subsubsection{Previsión de potencia}

La potencia total prevista para la instalación, usada para el cálculo del interruptor automático general se obtendrá, teniendo en cuenta los interruptores automáticos del cuadro general de mando y protección y un factor de corrección por simultaneidad 0,7:

$$P_{total}=\left( \sum \sqrt 3 \cdot I_{automatico\ i}\cdot  {V_{linea}}\right)\cdot 0,7$$\

Se obtiene un valor de {\bfseries potencia total de 222kW} para los receptores a instalar.\\

Para el cálculo la previsión de potencia de cada circuito se multiplicará la potencia nominal demandada por cada receptor por el número de receptores a instalar. Para circuito de las cintas transportadoras se aplicará un factor de corrección de 1,5 a la potencia total. Para el resto de receptores el factor de corrección será 1. \\

La previsión de potencia de cada circuito, así como el número de receptores a instalar y su distribución en los cuadros se encuentran detallados en el apartado \ref{sec:Planos} (Planos). Los cálculos se encuentran detallados en el apartado \ref{sec:Calculos} (Cálculos). 

\subsubsection{Cuadro General de Mando y Protección y cuadros secundarios}

La instalación eléctrica partirá del Cuadro General de Mando y Protección y se distribuirá entre dos cuadros secundarios como se detalla en el apartado \ref{sec:Planos} (Planos). La envolvente del Cuadro General de Mando y Protección es existente y se reutilizará de la instalación anterior. Las características de las envolventes de los cuadros secundarios vendrán detallados en el apartado \ref{sec:Presupuesto} (Presupuesto)\\

Los dispositivos generales e individuales de mando y protección de serán, como mínimo: un Interruptor general automático de corte omnipolar por cuadro, que permita su accionamiento manual y que esté dotado de elementos de protección contra sobrecarga y cortocircuitos, interruptores automáticos individuales para cada circuito con las mismas características que el interruptor general automático y un interruptor diferencial por cada 5 circuitos como mínimo. La sensibilidad de los interruptores diferenciales responderá a lo señalado en la Instrucción ITC-BT-24. Todas los Interruptores Automáticos Magnetotérmicos serán de corte omnipolar, tipo de curva C y poseerán un poder de corte de 6 kA. Se seleccionarán protecciones con un poder de corte de 6 kA al ser más económicos que los de poder de corte 4,5 kA que se exigen como mínimo por normativa.\\

La elección calibre de las protecciones de los interruptores automáticos individuales vendrá determinada por la potencia máxima demandada por cada receptor y la intensidad máxima admisible por los conductores que lo alimentan, siguiendo el criterio:\

$$ I_{MAX\ demandada\ receptor}\leq I_{AUTOMATICO}\leq I_{MAX\ admisible\ conductores}$$\

La distribución de los cuadros, así como los calibres de las protecciones y los circuitos a los que alimentan se encuentra detallado en el apartado \ref{sec:Planos} (Planos). Los cálculos de los calibres de los interruptores automáticos vendrán justificados en el apartado \ref{sec:Calculos} (Cálculos)\\

Los elementos a instalar en el cuadro vienen detallados en el \ref{sec:Presupuesto} (Presupuesto)]\

\subsubsection{Instalación eléctrica de la oficina}

Es la parte de la instalación eléctrica que partiendo del cuadro de de la oficina, enlaza con los receptores de iluminación, tomas de corriente de uso general, tomas de corriente de baño y tomas de corriente SAI.\\

Está regulada en las ITC-BT 25 e ITC-BT 26 del R.E.B.T. donde se especifican las prescripciones generales de instalación y calibres de los tubos a utilizar en cada circuito. Los conductores utilizados para estos circuitos serán de cobre unipolares con aislamiento de cobre y una tensión asignada de 450/750V (H07Z1-K) siendo no propagadores del incendio y con emisión de humos y opacidad reducida. La instalación de estos conductores realizará bajo tubo curvable de 2 capas en montaje empotrado en pared de mampostería o bajo falso techo, donde proceda.\\

\pagebreak

La elección de las secciones de los conductores vendrá determinada por la potencia máxima demandada por cada receptor y la intensidad máxima admisible por los conductores que lo alimentan, siguiendo siguiendo los siguientes criterios:\\

\underline{\bfseries Criterio de caída de tensión:}\

La caída de tensión sera como máximo del 3\% en los circuitos de alumbrado y un 5\% en el resto de instalaciones. Esta caída de tensión se calculará para una intensidad de funcionamiento del circuito igual a la intensidad nominal del interruptor automático de dicho circuito y para una distancia correspondiente a la del punto de utilización mas alejado del origen de la instalación. El valor de la caída de tensión podrá compensarse entre el de la instalación eléctrica de la oficina y el de la derivación del Cuadro General de Mando y Protección al cuadro de la oficina, de forma que la caída de tensión total sea inferior a la suma de los valores límite especificados para ambas. Se calculará usando las siguientes fórmulas:\\

Para circuitos de corriente alterna monofásica:\

$$ S=\frac{2\cdot \rho \cdot L_o	\cdot I\cdot cos(\varphi)}{\Delta V}$$\

Para circuitos de corriente alterna trifásica:\

$$ S=\frac{\sqrt{3} \cdot \rho \cdot L_o \cdot I\cdot cos(\varphi)}{\Delta V}$$\

Siendo $\Delta V$ la caída de tensión en voltios, $cos(\varphi)$ el factor de potencia activa, $L_o$ la longitud del cable en metros y $\rho$ la resistividad del conductor en $\Omega mm^2$\\


\underline{\bfseries Criterio de Intensidad máxima admisible:}\

Las Intensidades Máximas Admisibles de los conductores se regirán en su totalidad por lo indicado en la norma UNE 20460-5-523.\\

\underline{\bfseries Criterio de Intensidad de cortocircuito:}\

Se regirán en su totalidad por lo indicado en la ITC-BT 17. Se toma el defecto fase-neutro como el más desfavorable y se considera despreciable la reactancia inductiva de los conductores. La resistencia de los conductores para el cálculo será a 20 ºC\\
\pagebreak

A continuación se muestran los parámetros calculados para cada circuito, utilizando los criterios indicados, cuyos cálculos justificativos se pueden consultar en el apartado \ref{sec:Calculos} (Cálculos) y que se instalarán en los cuadros como se indica en el apartado  \ref{sec:Planos} (Planos).

\begin{figure}[!htb]
\centering
\includegraphics[width=1\textwidth]{Memoria/cuad1.png}
\caption{Elementos de la instalación en el CGMP}
\label{fig:digraph}
\end{figure}

\pagebreak

\begin{figure}[!htb]
\centering
\includegraphics[width=1\textwidth]{Memoria/cuad2.png}
\caption{Elementos de la instalación en el Cuadro de oficina}
\label{fig:digraph}
\end{figure}

\pagebreak

\begin{figure}[!htb]
\centering
\includegraphics[width=1\textwidth]{Memoria/cuad3.png}
\caption{Elementos de la instalación en el Cuadro de fábrica}
\label{fig:digraph}
\end{figure}

\pagebreak

Las instalación eléctrica de la oficina, al tratarse de un lugar de pública concurrencia y local de trabajo deberá cumplir las condiciones establecidas en la ITC-BT-28: \\

\begin{itemize}
\item El cuadro general de distribución deberá colocarse en el punto más próximo posible a la entrada de la acometida o derivación individual y se colocará junto o sobre él, los dispositivos de mando y protección establecidos en la instrucción ITC-BT-17. Cuando no sea posible la instalación del cuadro general en este punto, se instalará en dicho punto un dispositivo de mando y protección.\\

\item Del citado cuadro general saldrán las líneas que alimentan directamente los aparatos receptores o bien las líneas generales de distribución a las que se conectará mediante cajas o a través de cuadros secundarios de distribución los distintos circuitos alimentadores. Los aparatos receptores que consuman más de 16 amperios se alimentarán directamente desde el cuadro general o desde los secundarios.

\item El cuadro general de distribución e, igualmente, los cuadros secundarios, se instalarán en locales lugares o recintos a los que no tenga acceso el público y que estarán separados de los locales donde exista un peligro acusado de incendio o de pánico (cabinas de proyección, escenarios, salas de público, escaparates, etc.), por medio de elementos a prueba de incendios y puertas no propagadoras del fuego. Los contadores podrán instalarse en otro lugar, de acuerdo con la empresa distribuidora de energía eléctrica, y siempre antes del cuadro general.

\item En el cuadro general de distribución o en los secundarios se dispondrán dispositivos de mando y protección contra sobreintensidades, cortocircuitos y contactos indirectos para cada una de las líneas generales de distribución, y las de alimentación directa a receptores. Cerca de cada uno de los interruptores del cuadro se colocará una placa indicadora del circuito al que pertenecen.

\item En las instalaciones para alumbrado de locales o dependencias donde se reúna público, el número de líneas secundarias y su disposición en relación con el total de lámparas lámparas a alimentar, deberá ser tal que el corte de corriente en una cualquiera de ellas no afecte a más de la tercera parte del total de lámparas lámparas instaladas en los locales o dependencias que se iluminan alimentadas por dichas líneas. Cada una de estas líneas estarán protegidas en su origen contra sobrecargas, cortocircuitos, y si procede contra contactos indirectos.

\end{itemize}

Las canalizaciones serán de tubo corrugado para el montaje empotrado. Estos tipos de tubos serán antiinflamables y por tanto, no propagadores de la llama. Los angulos de curvatura no serán en ningún momento inferiores a 90, con el fin de permitir el acceso a los conductores. Las canalizaciones deben realizarse según lo dispuesto en las ITC-BT-19 e ITC-BT-20 y estaran constituidas por:

\begin{itemize}
\item Conductores aislados, de tensión nominal no inferior a 450/750 V, colocados bajo tubos o canales protectores, preferentemente empotrados en especial en las zonas accesibles al público.

\item Conductores aislados, de tensión nominal no inferior a 450/750 V, con cubierta de protección, colocados en huecos de la construcción, totalmente construidos en materiales incombustibles de grado de resistencia al fuego incendio RF-120, como mínimo.

\item Los cables y sistemas de conducción de cables deben instalarse de manera que no se reduzcan las características de la estructura del edificio en la seguridad contra incendios.

\item Los cables eléctricos a utilizar en las instalaciones de tipo general y en el conexionado interior de cuadros eléctricos en este tipo de locales, tendrán propiedades especiales frente al fuego, siendo no propagadores del incendio y con emisión de humos y opacidad reducida. Los cables con características equivalentes a la norma UNE 21.123, partes 4 ó 5, o a la norma UNE 211002 (según la tensión asignada del cable) cumplen con esta prescripción.

\item Los cables eléctricos a utilizar en las instalaciones de tipo general y en el conexionado interior de cuadros eléctricos en este tipo de locales, serán no propagadores del incendio y con emisión de humos y opacidad reducida. Los cables con características equivalentes a las de la norma UNE 21.123 parte 4 ó 5; o a la norma UNE 211002 (según la tensión asignada del cable), cumplen con esta prescripción.

\item Los elementos de conducción de cables con características equivalentes a los clasificados como "no propagadores de la llama" de acuerdo con las normas UNE-EN 50085-1 y UNE-EN 50086-1, cumplen con esta prescripción.

\item Los cables eléctricos destinados a circuitos de servicios de seguridad no autónomos o a circuitos de servicios con fuentes autónomas centralizadas, deben mantener el servicio durante y después del incendio, siendo conformes a las especificaciones de la norma UNE-EN 50.200 y tendrán emisión de humos y gases tóxicos muy opacidad reducida. Los cables con características equivalentes a la norma UNE 21.123, apartado 3.4.6, cumplen con esta prescripción de emisión de humos y opacidad reducida.

\end{itemize}

Los conductores de la instalación deben ser fácilmente identificados, especialmente por lo que respecta a los conductores neutro y de protección. Esta identificación se realizará por los colores que presenten sus aislamientos. Cuando exista conductor neutro en la instalación o se prevea para un conductor de fase su pase posterior a conductor neutro, se identificarán éstos por el color azul claro. Al conductor de protección se le identificará por el doble color amarillo-verde. Todos los conductores de fase, o en su caso, aquellos para los que no se prevea su pase posterior a neutro, se identificarán por los colores marrón o negro. Solamente cuando se considere necesario identificar tres fases diferentes, podrá utilizarse el color gris.\\

No se utilizará un mismo conductor neutro para varios circuitos. Todo conductor debe poder seccionarse en cualquier punto de la instalación en el que se realice una derivación del mismo, utilizando un dispositivo apropiado, tal como un borne de conexión, de forma que permita la separación completa de cada parte del circuito del resto de la instalación. En ningún caso se permitirá la unión de conductores mediante conexiones y o derivaciones por simple retorcimiento o arrollamiento entre sí de los conductores, sino que deberá realizarse siempre utilizando bornes de conexión montados individualmente o constituyendo bloques o regletas de conexión. Siempre deberán realizarse en el interior de cajas de empalme y / o derivación (Según ITC-BT 19).\\

La distribución de los circuitos así como las secciones de los conductores a instalar aparecen reflejados en el apartado \ref{sec:Planos} (Planos) del presente proyecto. Los cálculos de las secciones de los conductores vendrán justificados en el apartado \ref{sec:Calculos} (Cálculos)\pagebreak

\underline{\bfseries Cálculos luminotécnicos:}

Siguiendo la guía de buenas prácticas del Ministerio de Empleo y Seguridad Social NTP 211: Iluminación de los centros de trabajo, se tendrán en cuenta las disposiciones mínimas de iluminación en cada zona dependiendo del trabajo que se va a realizar. Según el Artículo 28 de la O.G.S.H.T. se garantizarán como mínimo:

\begin{itemize}

\item 300 lux para la zona de oficinas 

\item 50 lux para la zona de fabricación y almacén

\end{itemize}

El estudio luminotécnico que justifica el número de luminarias instaladas, su número y distribución se puede consultar en el apartado \ref{sec:Estudio luminotecnico} Estudio Luminotécnico\\

\underline{\bfseries Elementos de la instalación eléctrica de la oficina:}\\


{\bfseries Tomas de corriente en los puestos de trabajo con sistema de canales salvacables de PVC Quintela:}

La instalación eléctrica que alimentará las tomas de corriente de los puestos de trabajo se instalarán bajo canales salvacables Quintela. Los mecanismos de las tomas de corriente se instalarán en torretas metálicas TTM/4 para cuatro mecanismos, una torreta por puesto.\\

{\bfseries Sistema de alimentación ininterrumpida (SAI) Tecnosai Amplon RT5K:}

Sistema de alimentación ininterrumpida destinado a la alimentación de los equipos informáticos por medio del circuito de Tomas SAI desde el Cuadro Oficina. Su potencia nominal será de 5kVA para una salida monofásica de 230V.\\ 

Al tratarse de un SAI tipo offline, al producirse un fallo en el suministro de tensión se producirá una conmutación en un tiempo inferior a 10ms, conectando la alimentación a las baterías. El conexionado de la conmutación se instalará en el Cuadro Oficina como se indica el diagrama correspondiente en el apartado \ref{sec:Planos} (Planos)\

La unidad SAI Tecnosat Amplon RT5K y el pack de baterías se ubicarán en un armario rack de tipo mural de 19'' con 18U en sala de control de la segunda planta de la zona de oficinas. Las baterías se distribuirán en dos bandejas extraibles en la parte inferior del armario, quedando instalada la unidad SAI (2U) en la parte superior del mismo. El armario rack deberá disponer de ranuras de ventilación natural.\\



{\bfseries El resto de elementos seleccionados para los circuitos de la zona de oficinas, al no ser necesaria una descripción detallada de sus características y montaje, vendrán eflejados en el apartado \ref{sec:Presupuesto} (Presupuesto)}

\subsubsection{Instalación elécrtica de la zona de fabricación y almacén}

Las disposiciones establecidas para la instalación eléctrica de la zona de fabricación y almacén serán las mismas que las indicadas en el apartado anterior, teniendo en cuenta las siguientes particularidades:\\

Desde el Cuadro General de Mando y Protección discurrirán dos canalizaciones generales hasta el Cuadro Fábrica. Una de ellas albergará los conductores de alimentación de los diferentes circuitos de potencia y la otra los buses de comunicaciones entre el PC industrial y la sala de control. La canalización de los conductores de potencia será una Bandeja perforada Unex 60x200 en U23X y la de los buses de comunicación una Bandeja lisa Unex 60x75 en U23X. Estas se colocarán a una altura aproximada de 3 metros con respecto del nivel del suelo, discurriendo por la pared como viene especificado en el apartado \ref{sec:Planos} (Planos)\\

De la bandeja perforada que albergará los conductores de potencia nacerán las derivaciones a los circuitos, estas serán de tubo de acero galvanizado de montaje superficial si la instalación es superficial en pared y bajo salvacables para las tomas que sean de aplicación en un punto de la planta de la zona de fabricación. Se instalará un salvacables independiente para guiar los cables de datos de los sensores y DeviceBus desde las máquinas hasta el PC industrial en la zona de fabricación. Las tomas de corriente de uso genera se instalarán en la superficie de la pared una altura de 30cm. En el resto de puntos de alimentación se instalarán tomas hembra IEC 60309, dejando un exceso de cable de 1m en cada punto, como se especifica en el apartado \ref{sec:Planos} (Planos).\\

Las luminarias de la zona de fabricación y almacén se colocarán suspendidas del techo con cable de acero de 8mm de diámetro. Para alimentarlas se instalará una bandeja Bandeja perforada Unex 60x200 en U23X por la pared desde el Cuadro General de Mando y Protección hasta una altura de 10 metros, desde donde discurrirá a lo largo de la nave suspendida por cables de acero en cables de acero de 8mm de diámetro, siendo el punto de suspensión cada 3 metros. Las luminarias se situarán a una altura de 10 metros con respecto del nivel del suelo.

\newpage


	

%%%%%%%%%%%%%%%%%%%%%%%%%%%%%%%%%%%%%%%%%%%%%%%%%%%%%%%%%%%%%%%%%%%%%%%%%%%%%%%%%%%%%%%%%%%%%%%%%%%%%%%%%%%%%%%%%%%%%	
\section{Sistema propuesto}

\subsection{Características del sistema}

El proceso productivo objeto del presente proyecto se realizará enteramente en la zona de fabricación de la nave industrial. La nave se divide de dos áreas diferenciadas: un zona de oficinas de dos plantas con una superficie útil por planta de 175$m^2$ y la zona industrial con una superficie útil de 1260$m^2$, siendo la superficie total 1610${m^2}$. La zona industrial se divide en tres zonas: zona de carga y descarga, zona de almacenamiento de materia prima, zona de almacenamiento del producto y zona de fabricación. La zona de oficinas consta de una recepción, una zona de oficina por planta, zona de personal de fábrica y vestuarios, centro de control y una zona de aseos por planta.
\\

Los camiones entrarían marcha atrás en la nave y los empleados procederían a descargar su contenido con la ayuda de carretillas elevadoras. Los palets con los sacos con la materia prima en polvo, así como los de los cartones para ensamblar las cajas se almacenarían en estanterías de palets situadas en la zona de almacenamiento de materia prima. 
\\

En la zona de fabricación se llevará a cabo todo el proceso de transformación del polvo a pastillas envasadas en cajas. Se situarán un horno de secado, una máquina de compresión, una máquina de verificación, un tambor de revestimiento, una máquina de formación de blisters y una última de embalaje, así como cintas para transportar los distintos elementos y dos brazos robóticos para paletizar el producto final. Estos palets se almacenan en estanterías cerca de la puerta en la zona de almacenamiento de producto.\\

Las máquinas de la zona de fabricación disponen de una serie de sensores y actuadores que serán controlados por PC Industrial a través de una tarjeta de adquisición de datos. Se realizará tanto la instalación física de las máquinas, su electrificación y control, como la instalación eléctrica de la zona de oficinas y zona industrial. Todos los detalles y características de los elementos y procesos se encuentran detallados en las siguientes secciones del presente documento.

\subsection{Proceso de fabricación}

El proceso de fabricación de los comprimidos seguirá la siguiente secuencia: \\

Se proporcionará alimentación a todas las máquinas de forma manual y secuencial en el cuadro de fábrica accionando los interruptores automáticos correspondientes. Una vez comprobado que todas las máquinas están alimentadas se procederá a iniciar el sistema de control por PC industrial, que realizará una comprobación previa de todos los sistemas. El proceso comenzará precalentando el horno a la temperatura adecuada, una señal de aviso visual y acústica avisará al operario de que el horno está listo para recibir la materia prima y procederá a llenar el depósito con la cantidad adecuada. Una vez lleno, el horno liberará el polvo uniformemente sobre una cinta transportadora que lo lo transporta por el horno a una velocidad de entre 30 y 200mm por minuto, en función del tamaño de las pastillas a fabricar. El horno lleva una mezcla de gases $N_2$ y $H_2$ a una temperatura de unos 1050ºC para eliminar cualquier tipo de humedad. Se pueden regular la temperatura del horno, la velocidad de la cinta interna y la cantidad de polvo que se dispersa, dependiendo de las especificaciones del producto. En caso de que el proceso requiera de algún tipo de modificación manual, el operario podrá pasar de modo automático a modo manual en el panel de la máquina.
\\

A la salida del horno, el polvo pasa a una máquina de compresión, donde es comprimido en forma de pastillas con la serigrafía correspondiente en relieve. El tamaño y forma de las pastillas se deberá especificar e insertar el molde correspondiente manualmente antes de comenzar todo el proceso. \\  

Acto seguido pasa por la máquina de comprobación. Esta primero limpia las pastillas con un flujo de aire a presión que elimina el polvo disperso que quede de la operación anterior. Tras ser colocadas en fila la máquina comprueba una a una todas las pastillas para verificar que ninguna se rompe por debajo de un umbral mínimo que garantice la integridad de la misma durante su tiempo de vida útil y uso adecuado. También verificará periódicamente que ninguna pastilla escogida aleatoriamente se rompa por encima de un umbral máximo, lo que la podría hacer peligrosa para el consumo. Las pastillas desechadas se trituran y depositan en un recipiente para que el operario lo recircule en el horno en la siguiente iteración y así minimizar el desperdicio de materia prima. \\

El resto de pastillas continuará hasta la máquina de revestimiento. Un depósito regulado acepta tabletas hasta llegar a un volumen específico. Una vez alcanzada la cantidad máxima aceptable se vierte a un tambor giratorio para comenzar el proceso. Ahi una pistola de pintura regulada a una presión de entre 0.29MPa y 0.39MPa rocía en pulsos cortos y rápidos a las pastillas. La mezcla utilizada es de agua con colorantes y sabor artificial, para dotar a las mismas de un color identificativo, sabor agradable y protección, al endurecer de esta forma la capa exterior. Las pastillas están rodando y siendo rociadas durante el tiempo necesario para recubrirlas completamente, variando éste en función del tamaño de las tabletas y de la densidad del agente protector.\\

El proceso continua con las tabletas llegando a la máquina de formación de blisters. En esta máquina se dejan caer pastillas por una cinta transportadora hasta llenar un depósito. La selladora colocará las pastillas en posición según la matriz utilizada. Paralelamente tira del rollo de plástico en lámina y lo coloca debajo de un cabezal móvil caliente. Al bajar, ésta deforma la lámina con unos terminales con forma de pastilla y crea los habitáculos donde inmediatamente inserta las tabletas. Acto seguido se le coloca una lámina de aluminio con el distintivo impreso en relieve y es sellado por calor al bajar el cabezal móvil en la siguiente iteración. Finalmente, la máquina hace avanzar todo de nuevo y con la última bajada de cabezal corta el conjunto de plástico y aluminio relleno y sellado, y este avanza a la salida, donde se  envía a la máquina de empaquetado.\\

Finalmente se transportan los blisters sellados por otra cinta transportadora a la máquina de empaquetado. Esta coge las cartulinas de las cajas que previamente ha cargado un operario, y las dobla para darles su forma final. Inmediatamente después coge una hoja de prospecto y lo pliega hasta darle su forma característica. Finalmente introduce el prospecto y las unidades de blisters que correspondan en la caja y la cierra. Las cajas con el producto final se dejan caer por una rampa hasta una mesa donde un brazo robótico ABB IRB 260 equipado con una cámara coge las cajas individuales y las coloca en cajas de lotes en un palet para ser recogido por un operario y almacenado en una estantería. Un segundo brazo robótico ABB 4400 cojerá un palet de una pila de palets vacíos y lo colocará en la posición del anterior para el siguiente ciclo.
\\

%%%%%%%%%%%%%%%%%%%%%%%%%%%%%%%%%%%%%%%%%%%%%%%%%%%%%%%%%%%%%%%%%%%%%%%%%%%%%%%%%%%%%%%%%%%%%%%%%%%%%%%%%%%%%%%%%%%%%
\subsection{Flujograma del proceso de fabrcación}

\begin{figure}[htp]
			\centering
			\includegraphics[width=15cm,height=20cm,keepaspectratio]{Planos/Flujograma.png}
			\caption{Diagrama de flujo del proceso de fabricación}
			\label{fig:testa}
	\end{figure}


%%%%%%%%%%%%%%%%%%%%%%%%%%%%%%%%%%%%%%%%%%%%%%%%%%%%%%%%%%%%%%%%%%%%%%%%%%%%%%%%%%%%%%%%%%%%%%%%%%%%%%%%%%%%%%%%%%%%%%%%%%%%%%%%
\newpage\section {Firmas de los ingenieros}
\vspace{5cm}
Fdo. Alvaro Ferrán Cifuentes
\vspace{5cm}\hspace{5cm}
Fdo. David Antón Sánchez

%%%%%%%%%%%%%%%%%%%%%%%%%%%%%%%%%%%%%%%%%%%%%%%%%%%%%%%%%%%%%%%%%%%%%%%%%%%%%%%%%%%%%%%%%%%%%%%%%%%%%%%%%%%%%%%%%%%%%
\newpage \section{Anexos}

\subsection{Estudio Básico de Seguridad y Salud:}
\label{sec:EBSS}
\subsubsection{Introducción}
\includegraphics[width=15cm,keepaspectratio]{Memoria/EBSS/1.png}
\subsubsection{Derecho a la protección frente a los riesgos laborales}
\includegraphics[width=15cm,keepaspectratio]{Memoria/EBSS/2.png}
\subsubsection{Principios de la acción preventiva}
\includegraphics[width=15cm,keepaspectratio]{Memoria/EBSS/3.png}
\subsubsection{Equipos de trabajo y medios de protección}
\includegraphics[width=15cm,keepaspectratio]{Memoria/EBSS/4.png}
\subsubsection{Información, consulta y participación de los trabajadores}
\includegraphics[width=15cm,keepaspectratio]{Memoria/EBSS/5.png}
\subsubsection{Formación de los trabajadores}
\includegraphics[width=15cm,keepaspectratio]{Memoria/EBSS/6.png}
\subsubsection{Medidas de emergencia}
\includegraphics[width=15cm,keepaspectratio]{Memoria/EBSS/7.png}
\subsubsection{Riesgo grave e inminente}
\includegraphics[width=15cm,keepaspectratio]{Memoria/EBSS/8.png}
\subsubsection{Documentación}
\includegraphics[width=15cm,keepaspectratio]{Memoria/EBSS/9.png}
\subsubsection{Obligaciones de los trabajadores en materia de prevención de riesgos}
\includegraphics[width=15cm,keepaspectratio]{Memoria/EBSS/10.png}

\newpage




\subsection{Anexo de cálculos justificativos:}
\label{sec:Calculos}

\subsubsection{Cálculos de la Instalación eléctrica: conductores y protecciones}


\includepdf[pages=-, scale=0.9]{Memoria/calc1.pdf}

\pagebreak

\includepdf[pages=-, scale=0.9]{Memoria/calc2.pdf}

\pagebreak

\includepdf[pages=-, scale=0.9]{Memoria/calc3.pdf}

\pagebreak

\subsubsection{Estudio luminotécnico}
\label{sec:Estudio luminotecnico}

\includepdf[pages=-, scale=0.9]{Memoria/lumi1.pdf}

\includepdf[pages=-, scale=0.9]{Memoria/lumi2.pdf}







\newpage
\subsection{Anexo de código}

\pagebreak

\subsubsection{Programación en RAPID del sistema de paletizado con robots ABB}

\begin{figure}[!htb]
\centering
\includegraphics[width=1\textwidth]{Datasheets/robotstudio.png}
\caption{Proyecto RobotStudio}
\label{fig:digraph}
\end{figure}

\lstinputlisting{Memoria/Robotstudio_code.cpp}

\pagebreak

\subsubsection{Código detección de cajas en C++ utilizando OpenCV con algoritmo SURF}
\lstinputlisting{Memoria/codigo.cpp}
\newpage










\subsection{Anexo de catálogos}

\pagebreak

\subsubsection{Máquina de secado de polvo Hengli HSA1508-0711NH }
\includegraphics[width=15cm,height=20cm,keepaspectratio]{Datasheets/1Horno.png} 
\newpage

\subsubsection{Maquina de compresión en tabletas Unimach YUJ-17BZ}
\includegraphics[width=15cm,height=20cm,keepaspectratio]{Datasheets/2MaquinaPrensado.png} 
\newpage

\subsubsection{Maquina verificación de dureza de tabletas Unimach YJX220B}
\includegraphics[width=5cm,height=4cm,keepaspectratio]{Datasheets/3Foto.png} 
\\
\includegraphics[width=15cm,height=20cm,keepaspectratio]{Datasheets/3MaquinaVerificacion.png} 
\newpage

\subsubsection{Maquina de revestimiento de tabletas Unimach BGB80}
\includegraphics[width=5cm,height=4cm,keepaspectratio]{Datasheets/4Foto.png} 
\\
\includegraphics[width=15cm,height=20cm,keepaspectratio]{Datasheets/4MaquinaRevestimiento.png} 
\newpage

\subsubsection{Maquina de sellado de blisters Unimach DPK-260H2}
\includegraphics[width=5cm,height=4cm,keepaspectratio]{Datasheets/5Foto.png} 
\\
\includegraphics[width=15cm,height=20cm,keepaspectratio]{Datasheets/5MaquinaBlisters.png} 
\newpage

\subsubsection{Maquina de empaquetado Huan Yuan HTZ200}
\includegraphics[width=15cm,height=20cm,keepaspectratio]{Datasheets/7MaquinaEmpaquetado.png} 
\newpage
\subsubsection{Cinta transportadora TEMPO}
\includegraphics[height=22cm,keepaspectratio]{Datasheets/11Cinta.png} 
\newpage
\subsubsection{Robot ABB IRB 260}
\hspace*{-2cm}
\includegraphics[page=2]{Datasheets/IRB-260.pdf}
\newpage
\subsubsection{Robot ABB IRB 4400}
\hspace*{-2cm}
\includegraphics[page=2]{Datasheets/IRB-4400.pdf}
\newpage

\subsubsection{Controlador ABB IRC5}
\hspace*{-2cm}
\includegraphics[page=2]{Datasheets/IRC5.pdf}
\newpage

\subsubsection{Tarjeta DeviceNet PCI Softing CAN-AC1-PCI/DN}
\hspace*{-2cm}
\includegraphics[page=2]{Datasheets/CAN012E2_201204_CAN-AC1_PCI_DN.pdf}
\newpage

\subsubsection{Cables DeviceBus}
\includegraphics[width=1\textwidth]{Datasheets/DeviceBus.png}
\newpage

\subsubsection{Tarjeta de adquisición de datos Eagle NET-26C}
\hspace*{0cm}
\includegraphics[scale=0.63]{Datasheets/10Tarjeta.png}
\newpage
\subsubsection{PC industrial Siemens SIMATIC IPC547}
\hspace*{0cm}
\includegraphics[width=5cm,height=4cm,keepaspectratio]{Datasheets/12Foto.png} 

\includegraphics[scale=0.63]{Datasheets/12PC.png}
\newpage

\subsubsection{Luminaria downlight 2x26W Philips Fugato Performance}
\hspace*{-2cm}
\includegraphics[page=1]{Datasheets/ph1.pdf}
\newpage
\hspace*{-2cm}
\includegraphics[page=2]{Datasheets/ph1.pdf}
\newpage

\subsubsection{Luminaria fluorescente 4x14W Philips TBS165}
\hspace*{-2cm}
\includegraphics[page=1]{Datasheets/ph2.pdf}
\newpage
\hspace*{-2cm}
\includegraphics[page=2]{Datasheets/ph2.pdf}
\newpage

\subsubsection{Luminaria 1x315W Philips Megalux 4ME350}
\hspace*{-2cm}
\includegraphics[page=1]{Datasheets/ph3.pdf}
\newpage
\subsubsection{Luminaria Philips}
\hspace*{-2cm}
\includegraphics[page=2]{Datasheets/ph3.pdf}
\newpage