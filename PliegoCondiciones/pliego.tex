%%%%%%%%%%%%%%%%%%%%%%%%%%%%%%%%%%%%%%%%%%%%%%%%%%%%%%%%%%%%%%%%%%%%%%%%%%%%%%%%%%%%%%%%%%%%%%%%%%%%%%%%%%%%%%%%%%%%%%%%%%%%%%%%
\newpage

\section {Pliego de condiciones técnicas}

\subsection{Objetivo del pliego}
El presente pliego comprende el conjunto de condiciones técnicas que deberán cumplir la maquinaria instalada y la instalación eléctrica necesarias para la correcta ejecución del presente proyecto ``Instalación eléctrica y automatización de una fábrica de complementos alimenticios". 



\subsection{Condiciones maquinaria}
%%%%%%%%%%%%%%%%%%%%CINTAS
\paragraph{Cintas transportadoras:}
		Las cintas transportadoras utilizadas deberán cumplir las siguientes condiciones:
		
		\begin{itemize}
			\item{Longitud:}\
			
			Las longitudes de las cintas a instalar deberán ser las siguientes: 1.2m (dos unidades), 2.2m, 4.5m (tres unidades). La desviación admitada es de 0.02m.  

			\item{Anchura:}\
			
			El ancho de las cintas a instalar debe ser de 0.63m.

			\item{Altura:}\
			
			La altura de las cintas a instalar debe ser de 1.20m, con una tolerancia admitida de 0.01m.
			
			\item{Velocidad:}\
			
			La velocidad de las cintas deberá ser constante y su valor deberá encontrarse entre 0.3m/s y 0.5m/s, pudiendo transportar como mínimo 50kg de material por hora. 
			
			\item{Tipo de cubierta: }\
			
			La cubierta de las cintas deberá estar adaptada para el contacto directo con componentes destinados para consumo humano. Su superficie deberá ser lavable y desinfectable.

			\item{Color: }\
			
			Todas las cintas deberán ser de color negro para obtener el máximo contraste entre el polvo o las pastillas y la cinta.

			\item{Potencia demandada máxima (del conjunto de cintas):} 3.4kW.
		
		\end{itemize}

%%%%%%%%%%%%%%%%%%%%Horno
\paragraph{Máquina de secado:}
		La máquina de secado de polvo deberá cumplir las siguientes condiciones:
		
		\begin{itemize}
			\item{Longitud:}\\
			La longitud de la máquina de secado deberá ser de 2.31m. 

			\item{Anchura:}\\
			El ancho de la máquina debe ser menor de 1.80m.
			
			\item{Altura:}\\
			La altura de las entradas y salidas de la máquina debe ser de 1.20m, con una tolerancia admitida de 0.01m.
			
			
			\item{Condiciones de funcionamiento: }\\
		
			El horno industrial deberá funcionar en modo continuo, siendo capaz de procesar como mínimo 50kg de polvo. En la entrada deberá tener un depósito capaz de almacenar como mínimo 200kg de materia prima y de dispensar de forma homogénea el polvo sobre la cinta interna. Deberá alcanzar una temperatura mínima de 980ºC.\\
			La cinta interior deberá circular a una velocidad adecuada para el correcto secado del material. Asimismo deberá tener una superficie de contacto resistente a la temperatura máxima del horno y cumplir con todas las normativas pertinentes sobre el transporte de alimentos.
			

			\item{Potencia demandada máxima :} 6kW.
		
		\end{itemize}

%%%%%%%%%%%%%%%%%%%%COMPRESION
\paragraph{Máquina de compresión:}
		La máquina de compresión de pastillas deberá cumplir las siguientes condiciones:
		
		\begin{itemize}
			\item{Longitud:}\\
			La longitud de la máquina de compresión deberá ser menor de 2.82m.   

			\item{Anchura:}\\
			El ancho de la máquina debe ser menor de 1.53m.
			
			\item{Altura:}\\
			La altura de las entradas y salidas de la máquina debe ser de 1.20m, con una tolerancia admitida de 0.01m.
			
			
			\item{Condiciones de funcionamiento: }\\
			La máquina de compresión deberá ser capaz de procesar como mínimo 50kg de polvo por hora y deberá estar fabricada en acero inoxidable propio para el tratado de alimentos.
			

			\item{Potencia demandada máxima :} 4kW.
		
		\end{itemize}


%%%%%%%%%%%%%%%%%%%%VERIFICACION
\paragraph{Máquina de verificación:}
		La máquina de verificación de dureza deberá cumplir las siguientes condiciones:
		
		\begin{itemize}
			\item{Longitud:}\\
			La longitud de la máquina de verificación deberá ser de 3.1m. 

			\item{Anchura:}\\
			El ancho de la máquina debe ser de 1.55m.
			
			\item{Altura:}\\
			La altura de las entradas y salidas de la máquina debe ser de 1.20m, con una tolerancia admitida de 0.01m.
			
			
			\item{Condiciones de funcionamiento: }\\
			La máquina debe ser capaz de inspeccionar 120000 pastillas por hora o 50kg por hora. Debe ser capaz de inspeccionar pastillas de mínimo 3 mm de grosor.

			\item{Potencia demandada máxima :} 0.6kW.
		
		\end{itemize}



%%%%%%%%%%%%%%%%%%%%REVISTIMIENTO
\paragraph{Máquina de revestimiento:}
		La máquina de revestimiento de pastillas deberá cumplir las siguientes condiciones:
		
		\begin{itemize}
			\item{Longitud:}\\
			La longitud de la máquina de revestimiento deberá ser de 2.61m. 

			\item{Anchura:}\\
			El ancho de la máquina debe ser menor de 3.31m.
			
			\item{Altura:}\\
			La altura de las entradas y salidas de la máquina debe ser de 1.20m, con una tolerancia admitida de 0.01m.
			La altura máxima de la máquina no debe sobrepasar los 2.5m de altura.
			
			
			\item{Condiciones de funcionamiento: }\\
			La máquina de revestimiento debe soportar una carga mínima de 50kg. Deberá tener una entrada y una salida separadas para garantizar el paso automático de una cinta transportadora a la siguiente. El tambor deberá girar a una velocidad comprendida entre 2.0 y 16 revoluciones por minuto.
			

			\item{Potencia demandada máxima :} 1.5kW.
		
		\end{itemize}



%%%%%%%%%%%%%%%%%%%%BLISTERS
\paragraph{Máquina de blisters:}
		La máquina de formación de blisters deberá cumplir las siguientes condiciones:
		
		\begin{itemize}
			\item{Longitud:}\\
			La longitud de la máquina de formación de blisters deberá ser de 4.79m.  

			\item{Anchura:}\\
			El ancho de la máquina debe ser de 1.53m.
			
			\item{Altura:}\\
			La altura de las entradas y salidas de la máquina debe ser de 1.20m, con una tolerancia admitida de 0.01m.
			
			
			\item{Condiciones de funcionamiento: }\\
			La máquina de formación de blisters deberá posicionar las pastillas que le son suministradas automáticamente en una matriz del tamaño adecuado para su posterior empaquetado. Los polímeros utilizados para el embalaje de las pastillas deberán ser conformes con las normativas pertinentes respecto a la producción y manipulación de alimentos. Del mismo modo el aluminio utilizado para sellar las cápsulas deberá estar preparado para cumplir dichas normativas. 

			

			\item{Potencia demandada máxima :} 2.1kW.
		
		\end{itemize}		


%%%%%%%%%%%%%%%%%%%%EMPAQUETADO
\paragraph{Máquina de empaquetado:}
		La máquina de empaquetado deberá cumplir las siguientes condiciones:
		
		\begin{itemize}
			\item{Longitud:}\\
			La longitud de la máquina de empaquetado deberá ser de 4.1m. 

			\item{Anchura:}\\
			El ancho máximo de la máquina debe ser de 1.88m, siendo el ancho útil de paso de pastillas de 1.53m.
			
			\item{Altura:}\\
			La altura de las entradas y salidas de la máquina debe ser de 1.20m, con una tolerancia admitida de 0.01m.
			
			
			\item{Condiciones de funcionamiento: }\\
			La máquina debe plegar y sellar los cartones precortados y formar con ellos las cajas. Seguidamente deberá instertar un blister de pastillas y plegar e insertar en la caja el folleto explicativo correspondiente. Todo esto deberá hacerse con una frecuencia de al menos 200 cajas por minuto.
			

			\item{Potencia demandada máxima :} 3.4kW.
		
		\end{itemize}



%%%%%%%%%%%%%%%%%%%%%%%%%%%%%%%%%%%%%%%%%%%%%%%%%%%%%%%%%%%%%%%%%%%%%%%%%%%%%%%%%%%%%%%%%%%%%%%%%%%%%%%%%%%%%%%%%%%%%%%%%%%%%%%
\newpage

\subsection{Condiciones instalación eléctrica}

\subsubsection{Realización de la instalación}

Todas las instalaciones que comprenden este proyecto se realizarán cumpliendo las condiciones especificadas en los apartados \ref{sec:Memoria} (Memoria), \ref{sec:Calculos} (Cálculos), \ref{sec:Planos} (Planos) y \ref{sec:Presupuesto} (Presupuesto) del presente proyecto, ateniéndose siempre a las buenas prácticas constructivas y de presentación. En el apartado \ref{sec:Presupuesto} (Presupuesto) se indicarán los modelos de equipos y materiales a instalar. \\

Se cumplirá con las normas de seguridad especificadas en el apartado \ref{sec:EBSS} (Estudio Básico de Seguridad Salud.
La situación y detalles de instalación de cada parte de la instalación eléctrica vendrán especificados en el apartado \ref{sec:Planos} (Planos). La instalación se realizará acorde a la normativa vigente en materia de calidad y seguridad, tanto para el trabajador como para el usuario final de la instalación y seguirán las indicaciones del director de la obra en todo momento pudiendo plantearle cualquier consulta que sea necesaria para la correcta realización del presente proyecto.\\

Las instalaciones eléctricas se realizarán acorde a las prescripciones que, para cada una de las partes que la componen, se especifican en las correspondientes ITC-BT del R.E.B.T. Durante la ejecución de la instalación, todo trabajo se planteará por anticipado y cualquier corte, roza ó perforación que sea necesario realizar, se consultará con la suficiente antelación y se hará únicamente con la autorización previa de la Dirección y de conformidad con sus instrucciones. Todos los extremos de tubería abierta instalada, se protegerá con tapones ó cualquier otro método durante el tiempo que dure la obra. El reparto de las fases se hará equilibrando las cargas entre L1, L2 y L3 para conseguir que tengan el mejor reparto posible.\\

La instalación eléctrica se entenderá terminada cuando se hayan realizado las pruebas finales de aislamiento, continuidad en circuitos, reparto de cargas, así como puesta en marcha y prueba en carga real, es decir, alimentado los equipos mecánicos, alumbrado, etc.\\

Los dispositivos generales e individuales de mando y protección de serán, como mínimo: un Interruptor general automático de corte omnipolar por cuadro, que permita su accionamiento manual y que esté dotado de elementos de protección contra sobrecarga y cortocircuitos, interruptores automáticos individuales para cada circuito con las mismas características que el interruptor general automático y un interruptor diferencial por cada 5 circuitos como mínimo. La sensibilidad de los interruptores diferenciales responderá a lo señalado en la Instrucción ITC-BT-24. Todas los Interruptores Automáticos Magnetotérmicos serán de corte omnipolar, tipo de curva C y poseerán un poder de corte de 6 kA. Se seleccionarán protecciones con un poder de corte de 6 kA al ser más económicos que los de poder de corte 4,5 kA que se exigen como mínimo por normativa.\\


La caída de tensión sera como máximo del 3\% en los circuitos de alumbrado y un 5\% en el resto de instalaciones. Esta caída de tensión se calculará para una intensidad de funcionamiento del circuito igual a la intensidad nominal del interruptor automático de dicho circuito y para una distancia correspondiente a la del punto de utilización mas alejado del origen de la instalación. El valor de la caída de tensión podrá compensarse entre el de la instalación eléctrica de la oficina y el de la derivación del Cuadro General de Mando y Protección al cuadro de la oficina, de forma que la caída de tensión total sea inferior a la suma de los valores límite especificados para ambas.\\


Las instalación eléctrica de la oficina, al tratarse de un lugar de pública concurrencia y local de trabajo deberá cumplir las condiciones establecidas en la ITC-BT-28: \\

\begin{itemize}
\item El cuadro general de distribución deberá colocarse en el punto más próximo posible a la entrada de la acometida o derivación individual y se colocará junto o sobre él, los dispositivos de mando y protección establecidos en la instrucción ITC-BT-17. Cuando no sea posible la instalación del cuadro general en este punto, se instalará en dicho punto un dispositivo de mando y protección.\\

\item Del citado cuadro general saldrán las líneas que alimentan directamente los aparatos receptores o bien las líneas generales de distribución a las que se conectará mediante cajas o a través de cuadros secundarios de distribución los distintos circuitos alimentadores. Los aparatos receptores que consuman más de 16 amperios se alimentarán directamente desde el cuadro general o desde los secundarios.

\item El cuadro general de distribución e, igualmente, los cuadros secundarios, se instalarán en locales lugares o recintos a los que no tenga acceso el público y que estarán separados de los locales donde exista un peligro acusado de incendio o de pánico (cabinas de proyección, escenarios, salas de público, escaparates, etc.), por medio de elementos a prueba de incendios y puertas no propagadoras del fuego. Los contadores podrán instalarse en otro lugar, de acuerdo con la empresa distribuidora de energía eléctrica, y siempre antes del cuadro general.

\item En el cuadro general de distribución o en los secundarios se dispondrán dispositivos de mando y protección contra sobreintensidades, cortocircuitos y contactos indirectos para cada una de las líneas generales de distribución, y las de alimentación directa a receptores. Cerca de cada uno de los interruptores del cuadro se colocará una placa indicadora del circuito al que pertenecen.

\item En las instalaciones para alumbrado de locales o dependencias donde se reúna público, el número de líneas secundarias y su disposición en relación con el total de lámparas lámparas a alimentar, deberá ser tal que el corte de corriente en una cualquiera de ellas no afecte a más de la tercera parte del total de lámparas lámparas instaladas en los locales o dependencias que se iluminan alimentadas por dichas líneas. Cada una de estas líneas estarán protegidas en su origen contra sobrecargas, cortocircuitos, y si procede contra contactos indirectos.

\end{itemize}

Las canalizaciones serán de tubo corrugado para el montaje empotrado. Estos tipos de tubos serán antiinflamables y por tanto, no propagadores de la llama. Los angulos de curvatura no serán en ningún momento inferiores a 90, con el fin de permitir el acceso a los conductores. Las canalizaciones deben realizarse según lo dispuesto en las ITC-BT-19 e ITC-BT-20 y estaran constituidas por:

\begin{itemize}
\item Conductores aislados, de tensión nominal no inferior a 450/750 V, colocados bajo tubos o canales protectores, preferentemente empotrados en especial en las zonas accesibles al público.

\item Conductores aislados, de tensión nominal no inferior a 450/750 V, con cubierta de protección, colocados en huecos de la construcción, totalmente construidos en materiales incombustibles de grado de resistencia al fuego incendio RF-120, como mínimo.

\item Los cables y sistemas de conducción de cables deben instalarse de manera que no se reduzcan las características de la estructura del edificio en la seguridad contra incendios.

\item Los cables eléctricos a utilizar en las instalaciones de tipo general y en el conexionado interior de cuadros eléctricos en este tipo de locales, tendrán propiedades especiales frente al fuego, siendo no propagadores del incendio y con emisión de humos y opacidad reducida. Los cables con características equivalentes a la norma UNE 21.123, partes 4 ó 5, o a la norma UNE 211002 (según la tensión asignada del cable) cumplen con esta prescripción.

\item Los cables eléctricos a utilizar en las instalaciones de tipo general y en el conexionado interior de cuadros eléctricos en este tipo de locales, serán no propagadores del incendio y con emisión de humos y opacidad reducida. Los cables con características equivalentes a las de la norma UNE 21.123 parte 4 ó 5; o a la norma UNE 211002 (según la tensión asignada del cable), cumplen con esta prescripción.

\item Los elementos de conducción de cables con características equivalentes a los clasificados como "no propagadores de la llama" de acuerdo con las normas UNE-EN 50085-1 y UNE-EN 50086-1, cumplen con esta prescripción.

\item Los cables eléctricos destinados a circuitos de servicios de seguridad no autónomos o a circuitos de servicios con fuentes autónomas centralizadas, deben mantener el servicio durante y después del incendio, siendo conformes a las especificaciones de la norma UNE-EN 50.200 y tendrán emisión de humos y gases tóxicos muy opacidad reducida. Los cables con características equivalentes a la norma UNE 21.123, apartado 3.4.6, cumplen con esta prescripción de emisión de humos y opacidad reducida.

\end{itemize}

Los conductores de la instalación deben ser fácilmente identificados, especialmente por lo que respecta a los conductores neutro y de protección. Esta identificación se realizará por los colores que presenten sus aislamientos. Cuando exista conductor neutro en la instalación o se prevea para un conductor de fase su pase posterior a conductor neutro, se identificarán éstos por el color azul claro. Al conductor de protección se le identificará por el doble color amarillo-verde. Todos los conductores de fase, o en su caso, aquellos para los que no se prevea su pase posterior a neutro, se identificarán por los colores marrón o negro. Solamente cuando se considere necesario identificar tres fases diferentes, podrá utilizarse el color gris.\\

No se utilizará un mismo conductor neutro para varios circuitos. Todo conductor debe poder seccionarse en cualquier punto de la instalación en el que se realice una derivación del mismo, utilizando un dispositivo apropiado, tal como un borne de conexión, de forma que permita la separación completa de cada parte del circuito del resto de la instalación. En ningún caso se permitirá la unión de conductores mediante conexiones y o derivaciones por simple retorcimiento o arrollamiento entre sí de los conductores, sino que deberá realizarse siempre utilizando bornes de conexión montados individualmente o constituyendo bloques o regletas de conexión. Siempre deberán realizarse en el interior de cajas de empalme y / o derivación (Según ITC-BT 19).\\

\newpage

\section {Pliego de procesos de ejecución}

\subsection{Objetivo del pliego}	
El presente pliego comprende el conjunto de procesos de ejecución que se deberán cumplir al transportar e instalar los componentes necesarios para la correcta ejecución del presente proyecto ``Instalación eléctrica y automatización de una fábrica de complementos alimenticios".
\\
Todos los procesos deberán regirse por los reglamentos correspondientes y deberán realizarse siguiendo el espíritu y la correcta interpretación del presente proyecto. Asimismo, todos los elementos utilizados deberán estar de acuerdo con lo estipulado en el pliego de condiciones técnicas.

\subsection{Transporte de elementos}
Todos los elementos deberán ser transportados en su embalaje original si existiera y se deberán tomar todas las medidas necesarias para garantizar la integridad de todos los elementos y evitar cualquier tipo de daño durante el transporte. 
\\
Los elementos deberán ser almacenados hasta su instalación en una nave o estructura similar debidamente acondicionada para evitar posibles daños.

\subsection{Colocación y conservación de los elementos}
La colocación de las máquinas deberá seguir la disposición establecida en el plano correspondiente. El traslado por la nave deberá realizarse de manera a no obstruir el paso al resto de máquinas y con el cuidado y herramientas necesarios para evitar ocasionar daños de ningún tipo.\\

Todos los equipos y materiales se instalarán de acuerdo con las indicaciones del fabricante, siempre y cuando cumplan con la normativa vigente y las indicaciones de la Dirección de obra. Serán de la mejor calidad y se adaptarán a las tendencias y especificaciones de diseño actuales. En el caso de que los materiales seleccionados no se encuentren disponibles en el mercado por cualquier motivo, se seleccionarán otros de igual o superior calidad y características técnicas. Los equipos se colocarán en los espacios asignados, dejando un espacio razonable de acceso para su uso y reparación.\\


Los elementos contenidos en la obra, ya sea acopiados o instalados, son responsabilidad del Contratista hasta la recepción provisional de la instalación. En consecuencia, dispondrá los medios necesarios para su protección, tanto para evitar deterioros como desapariciones. Deberán protegerse los materiales contra golpes y humedades. Las aberturas de conexión de aparatos y equipos, al igual que los extremos de los tubos, permanecerán tapadas y protegidas hasta su montaje. Se tendrá un cuidado especial con los materiales más frágiles y delicados, como aparatos de control y regulación, materiales aislantes, etc., que se mantendrán especialmente protegidos. \\

\subsection{Conexionado de máquinas}
Todos los conexionados eléctricos deberán seguir la disposición establecida en el plano correspondiente del apartado \ref{sec:Planos} (Planos).  \\
Todo el cableado deberá ir anclado al suelo y cubierto por pasacables para su organización e identificación así como para evitar posibles accidentes.


%%%%%%%%%%%%%%%%%%%%%%%%%%%%%%%%%%%%%%%%%%%%%%%%%%%%%%%%%%%%%%%%%%%%%%%%%%%%%%%%%%%%%%%%%%%%%%%%%%%%%%%%%%%%%%%%%%%%%%%%%%%%%%%%
\newpage\section {Firmas de los ingenieros}
\vspace{5cm}
Fdo. Alvaro Ferrán Cifuentes
\vspace{5cm}\hspace{5cm}
Fdo. David Antón Sánchez
